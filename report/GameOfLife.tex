%% Based on a TeXnicCenter-Template by Gyorgy SZEIDL.
%%%%%%%%%%%%%%%%%%%%%%%%%%%%%%%%%%%%%%%%%%%%%%%%%%%%%%%%%%%%%

%----------------------------------------------------------
%
\documentclass{report}%
%
%----------------------------------------------------------
% This is a sample document for the standard LaTeX Report Class
% Class options
%       --  Body text point size:
%                        10pt (default), 11pt, 12pt
%       --  Paper size:  letterpaper (8.5x11 inch, default)
%                        a4paper, a5paper, b5paper,
%                       legalpaper, executivepaper
%       --  Orientation (portrait is the default):
%                       landscape
%       --  Printside:  oneside (default), twoside
%       --  Quality:    final (default), draft
%       --  Title page: titlepage, notitlepage
%       --  Columns:    onecolumn (default), twocolumn
%       --  Start chapter on left:
%                       openright(no), openany (default)
%       --  Equation numbering (equation numbers on right is the default)
%                       leqno
%       --  Displayed equations (centered is the default)
%                       fleqn (flush left)
%       --  Open bibliography style (closed bibliography is the default)
%                       openbib
% For instance the command
%          \documentclass[a4paper,12p,leqno]{report}
% ensures that the paper size is a4, fonts are typeset at the size 12p
% and the equation numbers are on the left side.
%
\usepackage{amsmath}%
\usepackage{amsfonts}%
\usepackage{amssymb}%
\usepackage{graphicx}
\usepackage{url}
%----------------------------------------------------------

%----------------------------------------------------------
\begin{document}

\title{Game of Life 2D}
\author{Federica Amato, Andrea Bonandin}
\date{February 2017}
\maketitle
\tableofcontents

%\part{The First Part}

\chapter{Introduction}
In the 1940s, the physicist and mathematician John von Neumann began working
on the solution of the mystery of self-reproduction as employed in biological systems. He was
interested in the use of computing devices to model the complex behavior of organisms.
Von Neumann was determined to answer the question, "What kind of logical organization is sufficient for
an automaton to be able to reproduce itself?" He proposed a solution to this question in
his book Theory of Self-Reproducing Automata (completed and published in 1966).

\noindent Since then, cellular automata have been used in a variety of ways and have taken on multiple
definitions, but it is generally agreed upon that all cellular automata consist of a collection of
'colored' cells on a grid of specified shape, that evolves through a number of discrete time steps
according to a set of rules based on the states of neighboring cells. As such, every cellular
automata consists of a certain type of cellular space and a transition rule. The cellular space can
be described in terms of any d-dimensional regular lattice of cells with finite boundary conditions.
Each cell has $k $number of states where $k$ is most commonly used as a positive integer ($k : k \in Z+$). The set of states is denoted $\sum$ , making $k = |\sum|$. 

\chapter{Conway's Game of Life}

In the 1970s, Cambridge Professor John Conway invented a 2-dimensional cellular automaton consisting of a Moore neighborhood that he called "life". In doing so, he was hoping to be able to study the macroscopic behaviors of a population.

\noindent The "game" is a zero-player game, meaning that its evolution is determined by its initial state, requiring no further input. One interacts with the Game of Life by creating an initial configuration and observing how it evolves, or, for advanced "players", by creating patterns with particular properties.

\section{Basic Rules}
Conway experimented with various transition rules and eventually decided on this setup:
\begin{itemize}

	\item $\sum = \{0, 1\}$ where a state of 0 represents a dead member of the population    and 1 represents an alive member of the population;
	\item The transition rule consists of either the death of a member (going from $1 \rightarrow 0$), birth of a member ($0 \rightarrow 1$), or no change ($0 \rightarrow 0$ or $1 \rightarrow 1$);
	\item With zero or one bordering member alive in a cells neighborhood, death occurs due to loneliness;
	\item With two or three members alive in the neighborhood, there is no change to an alive member;
	\item If a dead member's neighborhood consists of three live members, a new member is born ($0 \rightarrow 1$);
	\item If a live member's neighborhood consists of more than three live members, it will die due to overpopulation.
\end{itemize}













\appendix

\chapter{Serial Code}



\chapter{Parallel Code}



\begin{thebibliography}{9}
\bibitem {Life} Caleb Koch, "Regularity in Conway's Game of Life", 2015
\bibitem {wiki} Web, \url{https://en.wikipedia.org/wiki/Conway's_Game_of_Life}
\bibitem {stan} Web, \url{http://web.stanford.edu/~cdebs/GameOfLife}
\end{thebibliography}
\end{document}

\end{document}
